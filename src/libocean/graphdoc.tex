\documentstyle[a4]{article}

%%%% @(#)graphdoc.tex 1.9 10/31/94 Delft University of Technology

    % Leave some vertical space between paragraphs; No paragraph indent:
    %\setlength{\parskip}{0.6\baselineskip} \setlength{\parindent}{0cm}

    % When is it allowed to place a figure or a table:
    \renewcommand{\textfraction}{0.05}     \renewcommand{\topfraction}{0.95}
    \renewcommand{\bottomfraction}{0.95}

    % input this to enable arbitrary PostScript inclusion with \psfig:
    \input{psfig}

    % directory where to look for figures:
    \newcommand{\figdir}{/usr1/ocean/src/libocean/doc}

    % macro \fig inserts a centered figure:
    \newcommand{\fig}[1]{\centerline{\psfig{figure=\figdir/#1.ps}}}

    % Macro \figopt also inserts a PostScript figure, but this one requires
    % extra options, as in \figopt{psfile}{heigth=2.1in,width=3.5in}.
    %          !!!! Do not use spaces in the option list !!!!
    \newcommand{\figopt}[2]{\centerline{\psfig{figure=\figdir/#1.ps,#2}}}

    \newcommand{\file}[1]{{#1}}
    \newcommand{\class}[1]{{#1}}
    \newcommand{\type}[1]{{#1}}

%\psdraft

\begin{document}

\title{\sf Graph --- A set of C++ base classes for hypergraphs}
\author{\sf Paul Stravers\\ \sf Delft University of Technology}
\date{\sf 94/10/31}
\maketitle
\noindent\makebox[\textwidth]{\hrulefill}
\tableofcontents
\vspace*{3ex}
\noindent\makebox[\textwidth]{\hrulefill}
\section{Introduction}
The files \file{graph.h} and \file{graph.C} in the ocean class library define a
set of base classes that implement a hypergraph. A hypergraph differs from an
ordinary graph in the number of vertices that can be incident to an edge. In an
ordinary graph this number is always 2, in a hypergraph any number of vertices
can be incident to an edge.

In many applications, a vertex is part of more than one graph. As an example,
consider a digital circuit that we represent as a hypergraph. In this case
a vertex represents a module in the circuit, and the edges represent the wires
between the modules (Figure~\ref{f-sr-latch}). At the same time, however, the
vertices are also part of a hierarchy graph. In Figure~\ref{f-sr-latch} the
hierarchy graph specifies that the modules {\sl nand1, nand2\/} and {\sl inv\/}
are all children of the module {\sl sr-latch\/}. The circuit graph specifies
how these three children connect to each other. To complicate things even more,
the hierarchy graph is a directed graph but the circuit graph is undirected.

\begin{figure}[hbt]
\figopt{sr-latch}{width=1.0\textwidth}
\caption{\label{f-sr-latch}
A digital circuit represented as a hypergraph. The solid, undirected
edges represent the wires between the modules. The directed (dashed) edges
represent the hierarchy relationship.}
\end{figure}

With the hypergraph base class we can easily implement a graph like the one
in Figure~\ref{f-sr-latch}. The class allows to include a vertex in as many
graphs as we need. Edges can be both directed and undirected. There are
iterator classes that allow us to iterate through the members of a particular
graph in a systematic way.

\section{How to incorporate hypergraphs in your C++ program}
To use the hypergraph class library, include the line
\begin{verbatim}
  #include <graph.h>
\end{verbatim}
in the C++ program source. You must instruct the C++ compiler to search the
ocean include directory for header files. You also must tell the compiler to
link your program with the ocean class library {\tt \$MACHINE/libocean.a}. For
instance, if you are on a Hewlett Packard hp800 computer, the following Unix
commands are appropriate to compile and link myprog.C:
\begin{verbatim}
  % CC -I/usr/ocean/include -c myprog.C
  % CC -L/usr/ocean/lib/hp800 myprog.o -locean
\end{verbatim}
This assumes that the ocean tree starts in {\tt /usr/ocean}. Your system
administrator may have moved it to another place.

\section{The structure of the graph class}
A \class{graphDescriptor} is the ``box'' that holds a graph. The only thing a
\class{graphDescriptor} knows about is a \class{graphElement}. This element is
the common baseclass for objects from the classes \class{graphVertex} and
\class{graphEdge}.

You need objects from the class \class{graphTerminal} to ``glue'' a
\class{graphVertex} to a \class{graphEdge}. Each \class{graphTerminal} has a
\type{graphTermType} that identifies the type of the edge the vertex connects
to. For instance, the \class{graphVertex} {\sl sr-latch} in
Figure~\ref{f-sr-latch} is connected to three edges through three terminals of
type \type{EdgeToChild}. The vertices {\sl nand1, nand2} and {\sl inv\/} are
connected to the other side of these same three edges through terminals of type
\type{EdgeToFather}. Figure~\ref{f-graphglue} shows this in detail.

\begin{figure}[hbt]
\figopt{graphglue}{width=0.6\textwidth}
\caption{\label{f-graphglue}
\class{graphTerminal}s are the ``glue'' between vertices and edges.
The type of the terminal (\type{graphTermType}) tells the vertex what kind of
edge it connects to. This figure shows three directed edges (dashed lines) from
the vertex {\sl sr-latch} to its three children {\sl nand1, nand2} and {\sl
inv}.}
\end{figure}

Fortunately, most users of the graph classes never deal with the
\class{graphTerminal}s directly. Their creation and destruction is taken care
of automatically by the \class{graphEdge} class. Consequently, most users can
skip the Sections~\ref{s-terminal} and~\ref{s-terminal-itr} that deal with
\class{graphTerminal}s ---it is just an implementation detail.

\section{The \class{graphDescriptor}}
Before you can use graphs you must create a \class{graphDescriptor} object.
This object can contain objects of type \class{graphElement}, i.e. the vertices
and the edges. By making a \class{graphDescriptor} the {\em current}
graph descriptor, all newly created vertices and edges automatically become
part of this graph. Three methods exist to make a graph current. First is to pass
the constant {\tt SetCurrentGraph} as the second argument to the constructor.
The second method is calling the member function {\tt setCurrentGraph()}. The
third method is calling the member {\tt pushCurrentGraph()}. This function not
only sets the current graph, but also remembers (on a stack) what the
``current'' current graph is, so that you can restore to this state by calling
the function popCurrentGraph().  The public declarations appear below.

{\footnotesize \begin{verbatim}
  class graphDescriptor
  {
  public:
     graphDescriptor(char *name =NIL, int setCurrent =NIL);
     ~graphDescriptor();
     graphDescriptor *addElement(graphElement *);
     graphDescriptor *removeElement(graphElement *);
     void setCurrentGraph();    // make this descriptor ``current''
     void pushCurrentGraph();   // likewise, but undo with popCurrentGraph()
     void setName(char *);      // associates a string with this descriptor
     char *getName();           // returns a string associated with descriptor
     void print();              // print this graph to cout
  };
  
  // These are friend functions of class graphDescriptor:
  graphDescriptor *getCurrentGraph(); // return the current graph
  graphDescriptor *popCurrentGraph(); // undo last pushCurrentGraph()
\end{verbatim}}

\section{The \class{graphElement}}
The \class{graphElement} is the common base class for the \class{graphVertex}
and \class{graphEdge} classes. It defines a virtual member function
{\tt graphelementtype()} that returns the constant GenericGraphElementType. This
member function is later redefined by \class{graphVertex} to return the
constant VertexGraphElementType and by \class{graphEdge} to return
EdgeGraphElementType.

{\footnotesize \begin{verbatim}
  enum graphElementType
  {
     GenericGraphElementType,
     VertexGraphElementType, EdgeGraphElementType,
     AllGraphElementTypes
  };

  class graphElement : public Object
  {
  public:
     graphElement();
     ~graphElement();
     virtual void printElement() {cout << "<graphElement>" << flush;};
     virtual graphElementType graphelementtype() {return GenericGraphElementType;}
  };
\end{verbatim}}

\subsection{the \class{graphVertex}}
The \class{graphVertex} class is derived from \class{graphElement}. It has a
virtual member function {\tt vertexType()} returning an enum that
identifies the type of the vertex.  This is useful if we derive several vertex
classes from \class{graphVertex} and we want to be able to distinguish between.
The definition of this enum is conditional; we can define our own enum
\type{graphVertexType} and \#define {\tt GRAPHVERTEXTYPE\_DEFINED} before we
{\tt \#include <graph.h>}. In this way the member function {\tt
graphelementtype()} is as versatile as possible.

{\footnotesize \begin{verbatim}
  #ifndef GRAPHVERTEXTYPE_DEFINED
  enum graphVertexType {GenericVertex=0};
  #endif // GRAPHVERTEXTYPE_DEFINED

  class graphVertex : public graphElement
  {
  public:
     graphVertex();
     ~graphVertex();
     graphVertex *addToVertex(graphTerminal *newTerm); // init term yourself
     int hasEdgeTo(graphVertex *V, graphTermType T =GenericEdge);
     virtual graphVertexType vertexType() { return GenericVertex; }
     virtual void print();        // print the vertex to cout
     virtual void printElement(); // this and next from base class graphElement
     virtual graphElementType graphelementtype() {return VertexGraphElementType;}
  };
\end{verbatim}}

The function {\tt addToVertex()} glues a \class{graphTerminal} to the vertex.
It is the responsibility of the user to initialize this terminal properly (you
really have to know what you are doing, use with care).  The function {\tt
hasEdgeTo()} returns {\tt TRUE} if an edge from {\tt this} vertex to V exists.
Only edges of type T are considered.

\subsection{the \class{graphEdge}}
The \class{graphEdge} class is also derived from \class{graphElement}. Once
created, we can connect vertices to an edge using the function {\tt
addToEdge()} and remove them again using {\tt removeFromEdge()}.

{\footnotesize \begin{verbatim}
  class graphEdge : public graphElement
  {
  protected:
     virtual graphTerminal *newTerminal() {return new graphTerminal;}
  public:
     graphEdge();
     ~graphEdge();
     graphEdge     *addToEdge(graphVertex *, graphTermType =GenericEdge,
                              graphTerminal * =NIL);
     graphEdge     *addToEdge(graphTerminal *);     // initialize term yourself
     graphEdge     *removeFromEdge(graphVertex *);
     graphVertex   *otherSide(graphVertex *thisSide); // if edge has 2 vertices
     virtual void  print();
     virtual void  printElement(); // this and next from base class graphElement
     virtual graphElementType graphelementtype() {return EdgeGraphElementType;}
  };
\end{verbatim}}

As explained above, {\tt addToEdge()} needs a new \class{graphTerminal} in
order to connect a vertex to the edge. There are two ways to provide this
\class{graphTerminal}. First, we can supply it as an argument to the member
function {\tt addToEdge()}. Second, if we do not supply the third argument,
{\tt addToEdge()} calls the virtual protected member {\tt newTerminal()} to
create a \class{graphTerminal}. Optionally we can redefine it in a derived
class to create a different kind of terminal.

Note that there are two versions of {\tt addToEdge()}. The second one takes
only a \class{graphTerminal} as an argument and leaves the correct
initialization of this terminal to the user. This may sometimes be useful, for
instance to move a terminal from one edge to another edge

The member function {\tt otherSide()} has only a defined behavior if the
\class{graphEdge} connects to exactly two vertices. In that case it returns the
vertex that is not the vertex on thisSide.

The following programming example shows how to create a simple graph consisting
of two vertices connected by an edge.

{\footnotesize \begin{verbatim}
  #include <graph.h>
  graphDescriptor gdesc("example graph",SetCurrentGraph);
  graphEdge *e = new graphEdge;  // create an edge...
  // ...and add two vertices to this edge:
  e->addToEdge(new graphVertex)->addToEdge(new graphVertex);
\end{verbatim}}

A slightly more complicated example shows how to create a directed edge between
two vertices:

{\footnotesize \begin{verbatim}
  #include <graph.h>
  graphDescriptor gdesc("example graph",SetCurrentGraph);
  graphEdge *e = new graphEdge;  // create an edge...
  // ...and add two vertices to this edge:
  e->addToEdge(new graphVertex,DirectedEdgeOpen)
   ->addToEdge(new graphVertex,DirectedEdgeClosed);
\end{verbatim}}

\section{The \class{graphTerminal} and the \type{graphTermType}}
\label{s-terminal}
A \class{graphTerminal} ``glues'' a \class{graphVertex} to a \class{graphEdge}.
Most users will never have to worry about \class{graphTerminal}s, the
\class{graphEdge} members take care of the terminals automatically. For this
reason ---and because of laziness--- we do not discuss the \class{graphTerminal} in
this paper.  Have a look at the files graph.[hC]\ldots

As mentioned above, graph terminals have an associated \type{graphTermType}.
The enum \type{graphTermType} defines which terminal types are allowed. We can
redefine this enum and \#define {\tt GRAPHTERMTYPE\_DEFINED} before including {\tt
<graph.h>} in order to change or enlarge the set of allowable types. The
default definition of the enum is shown below.

{\footnotesize \begin{verbatim}
  #ifndef GRAPHTERMTYPE_DEFINED
  enum graphTermType
  {
     GenericEdge=0,
     EdgeToFather=1, EdgeToSibling=2, EdgeToChild=3,
     DirectedEdgeOpen=4, DirectedEdgeClosed=5,
     EdgeType1=11, EdgeType2=12, EdgeType3=13,
     EdgeType4=14, EdgeType5=15, EdgeType6=16
  };
  #endif // GRAPHTERMTYPE_DEFINED
\end{verbatim}}

Note that alternative definitions of \type{graphTermType} must at least include
the constant {\tt GenericEdge=0}.

\section{Iterators}
\label{s-iterators}
An iterator is an object that returns pointers to other objects in a systematic
way. For instance, an iterator might return all the vertices that connect to a
certain edge. Iteration does not destroy or modify the data structure that is
being iterated.

As the name suggests, an iterator only returns one object of a data structure
at a time. Repeatedly ``activating'' the iterator returns the succeeding
objects. All the iterators in the graph class have a similar user interface.
This interface consists of the two member functions {\tt initialize()} and {\tt
more()}. There is also an operator {\tt ()} to ``activate'' the iterator.
\begin{itemize}
\item
The member {\tt initialize()} resets the iterator to its initial state. After
resetting it, the iterator is ready for use. The constructor of an iterator
always calls the {\tt initialize()} function, and therefore always has the same
argument list as {\tt initialize()}.
\item
The operator {\tt ()} returns the next object from the iterated data structure,
or NIL if there are no more objects.
\item
The member {\tt more()} returns {\tt TRUE} if the next call to the operator
{\tt ()} would return an object. It returns {\tt NIL} if the next call to the
operator {\tt ()} would return {\tt NIL}. Thus, {\tt more()} tells you whether
there is anything left to iterate.
\end{itemize}
What happens if the user wants to modify or even delete objects in the data
structure that is being iterated? The iterators of the graph class library
contain internal state that at any time points to the next object to be
returned by the {\tt ()} operator. As a consequence, the only thing that a user
should never do is deleting or corrupting this ``next'' object. But it {\em is}
possible to modify or delete any other object, including the object that was
most recently returned by the {\tt ()} operator.

\subsection{An iterator that returns graph elements}
One important use of \class{graphElement}s is iterating all the members of a
graph, for instance with the purpose of initializing a flag or something
similar. The \class{graphElementIterator} class provides this functionality.

{\footnotesize \begin{verbatim}
  #ifndef GRAPHELEMENTTYPE_DEFINED
  enum graphElementType
  {
     GenericGraphElementType=0,
     VertexGraphElementType=1, EdgeGraphElementType=2,
     AllGraphElementTypes= -1
  };
  #endif // GRAPHELEMENTTYPE_DEFINED

  class graphElementIterator
  {
  public:
     graphElementIterator(graphDescriptor *,
                          graphElementType =AllGraphElementTypes);
     void initialize(graphDescriptor *,graphElementType =AllGraphElementTypes);
     graphElement *operator()();               // return the next graph element
     int more() {return currentelm!=NIL;}      // TRUE if any element left
  };
\end{verbatim}}

The following programming example shows how to use the iterator.

{\footnotesize \begin{verbatim}
  #include <graph.h>
  graphDescriptor gdesc("example graph", SetCurrentGraph);
  
       (...at this place we build a graph...)
  
  // Create an iterator that successively returns all the vertices in the graph:
  graphElementIterator nextElement(&gdesc,VertexGraphElementType);
  graphElement *e;
  while (e = nextElement())   // iterate trough the vertices of graph gdesc
     e->printElement();       // print each vertex in the graph
\end{verbatim}}

If you omit the second argument of the constructor, the constant
\type{AllGraphElementTypes} is selected, which means that the operator {\tt ()}
returns both vertices and edges.

\subsection{Iterators that return vertices}
The \class{graphVertexNeighborIterator} successively returns the neighbor
vertices of the vertex it was {\tt initialize()}d with, or of the vertex that
was passed as an argument to the constructor. You can ``filter'' the neighbors
by restricting the type of the terminals to something more specific than {\tt
genericEdge}.

{\footnotesize \begin{verbatim}
  class graphVertexNeighborIterator
  {
  public:
     graphVertexNeighborIterator(graphVertex *, graphTermType thisSide =GenericEdge,
                                 graphTermType otherSide =GenericEdge);
     void initialize(graphVertex *, graphTermType thisSide =GenericEdge,
                     graphTermType otherSide =GenericEdge);        // reset
     graphVertex *operator()();              // return the next neighbor vertex
     int more();                             // TRUE if anything left
  };
\end{verbatim}}

Use of this iterator object hides the implementation details of the graph, and
it helps in writing clean code. The following programming example prints all
the neighbors of the vertex vx, separated by commas and ended by a newline.

{\footnotesize \begin{verbatim}
  #include <graph.h>
  graphVertex *vx = new graphVertex;

       (...at this place we build a graph...)

  graphVertexNeighborIterator nextNeighbor(vx);   // create iterator object
  graphVertex *v;
  while (v = nextNeighbor())          // iterate trough the neighbors of vx
  {
     v->print();                             // print each neighbor to cout
     if (nextNeighbor.more())                // is this the last neighbor?
        cout << ", ";                        // no: print a comma
     else
        cout << "\n";                        // yes: print a newline
  }
\end{verbatim}}

The \class{graphEdgeVertexIterator} returns all the vertices that are incident
with (``connected to'') an edge. The public interface looks as follows:

{\footnotesize \begin{verbatim}
  class graphEdgeVertexIterator
  {
  public:
     graphEdgeVertexIterator(graphEdge *, graphTermType =GenericEdge);
     void initialize(graphEdge *, graphTermType =GenericEdge);
     graphVertex *operator()();         // return the next vertex in the edge
     int more() {return currentterminal!=NIL;} //TRUE if there is anything left
  };
\end{verbatim}}

\subsection{Iterators that return edges}
The \class{graphVertexEdgeIterator} allows us to iterate trough the edges that
connect to a vertex. In a similar way, the \class{graphVertexCommonEdgeIterator}
allows to iterate through the edges that are common to two distinct vertices.

{\footnotesize \begin{verbatim}
  class graphVertexEdgeIterator
  {
  public:
     graphVertexEdgeIterator(graphVertex *, graphTermType =GenericEdge);
     void initialize(graphVertex *, graphTermType =GenericEdge);     // reset
     graphEdge *operator()();                         // return the next edge
     int more() {return currentterminal!=NIL;}       // TRUE if anything left
  };
  
  class graphVertexCommonEdgeIterator
  {
  public:
     graphVertexCommonEdgeIterator(graphVertex *, graphVertex *,
                                   graphTermType =GenericEdge);
     void initialize(graphVertex *, graphVertex *,
                     graphTermType =GenericEdge);
     graphEdge *operator()();                  // return the next common edge
     int more() {return currentterminal!=NIL;}       // TRUE if anything left
  };
\end{verbatim}}

The following programming example shows how to iterate through all the directed
edges that point from vx1 to vx2:

{\footnotesize \begin{verbatim}
  #include <graph.h>
  graphVertex *vx1 = new graphVertex;
  graphVertex *vx2 = new graphVertex;

       (...at this place we build a directed graph...)

  graphVertexCommonEdgeIterator nextCmnEdge(vx1,vx2,DirectedEdgeOpen);
  graphEdge *e;
  while (e = nextCmnEdge())   // iterate the edges pointing from vx1 to vx2
     e->print();              // print the common, directed edge
\end{verbatim}}

\subsection{Iterators that return terminals}
\label{s-terminal-itr}
It sometimes is useful to iterate the terminals that belong to a certain
vertex, or to iterate the terminals that are contained in a certain edge. The
next two iterators provide this functionality.

{\footnotesize \begin{verbatim}
  class graphVertexTerminalIterator
  {
  public:
     graphVertexTerminalIterator(graphVertex *, graphTermType =GenericEdge);
     void initialize(graphVertex *, graphTermType =GenericEdge);
     graphTerminal *operator()();           // return the next terminal in the vertex
     int more() {return currentterminal!=NIL;} //TRUE if there is anything left
  };
  
  class graphEdgeTerminalIterator
  {
  public:
     graphEdgeTerminalIterator(graphEdge *, graphTermType =GenericEdge);
     void initialize(graphEdge *, graphTermType =GenericEdge);
     graphTerminal *operator()();             // return the next terminal in the edge
     int more() {return currentterminal!=NIL;} //TRUE if there is anything left
  };
\end{verbatim}}

The user can restrict the type of the terminals that the () operator returns to
one of the \type{graphTermType}s. The type \type{GenericEdge} matches all
available \type{graphTermType}s.

\end{document}
