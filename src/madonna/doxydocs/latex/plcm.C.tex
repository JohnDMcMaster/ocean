\subsection{plcm.C File Reference}
\label{plcm.C}\index{plcm.C@{plcm.C}}
{\tt \#include $<$stdlib.h$>$}\par
{\tt \#include $<$stdio.h$>$}\par
{\tt \#include $<$time.h$>$}\par
{\tt \#include $<$string.h$>$}\par
{\tt \#include "plcm.h"}\par
{\tt \#include "cluster.h"}\par
{\tt \#include "plane.h"}\par
{\tt \#include "prot\-Area.h"}\par
{\tt \#include $<$sys/types.h$>$}\par
\subsubsection*{Defines}
\begin{CompactItemize}
\item 
\#define {\bf TRANS\_\-DEBUG}\ 1
\end{CompactItemize}
\subsubsection*{Functions}
\begin{CompactItemize}
\item 
pid\_\-t {\bf getpid} (void)
\end{CompactItemize}


\subsubsection{Define Documentation}
\label{plcm.C_a0}
\index{plcm.C@{plcm.C}!TRANS_DEBUG@{TRANS\_\-DEBUG}}
\index{TRANS_DEBUG@{TRANS\_\-DEBUG}!plcm.C@{plcm.C}}
\paragraph{\setlength{\rightskip}{0pt plus 5cm}\#define TRANS\_\-DEBUG\ 1}\hfill



Definition at line 29 of file plcm.C.

\subsubsection{Function Documentation}
\label{plcm.C_a1}
\index{plcm.C@{plcm.C}!getpid@{getpid}}
\index{getpid@{getpid}!plcm.C@{plcm.C}}
\paragraph{\setlength{\rightskip}{0pt plus 5cm}pid\_\-t getpid (void)}\hfill



Referenced by {\bf Plcm::Plcm}().