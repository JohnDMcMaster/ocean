\subsection{matrix\-Int  Class Reference}
\label{matrixInt}\index{matrixInt@{matrix\-Int}}
{\tt \#include $<$matrix\-Int.h$>$}

Inheritance diagram for matrix\-Int:\begin{figure}[H]
\begin{center}
\leavevmode
\setlength{\epsfysize}{2cm}
\epsfbox{class_matrixint.eps}
\end{center}
\end{figure}
\subsubsection*{Public Methods}
\begin{CompactItemize}
\item 
{\bf matrix\-Int} (int dimension1, int dimension2)
\item 
{\bf matrix\-Int} (matrix\-Int\&)
\item 
{\bf $\sim$matrix\-Int} ()
\item 
int$\ast$ {\bf operator[$\,$]} (int i)
\item 
int {\bf dimension1} ()
\item 
int {\bf dimension2} ()
\item 
matrix\-Int\& {\bf exchange\-Rows} (int a, int b)
\item 
matrix\-Int\& {\bf exchange\-Columns} (int a, int b)
\item 
void {\bf print} ({\bf Transposition} =Not\-Transposed)
\end{CompactItemize}
\subsubsection*{Private Methods}
\begin{CompactItemize}
\item 
void {\bf create\-And\-Initialize} (int dimension1, int dimension2)
\end{CompactItemize}
\subsubsection*{Private Attributes}
\begin{CompactItemize}
\item 
int$\ast$$\ast$ {\bf thematrix}
\item 
int {\bf dim1}
\item 
int {\bf dim2}
\end{CompactItemize}


\subsubsection{Constructor \& Destructor Documentation}
\label{matrixInt_a0}
\index{matrixInt@{matrix\-Int}!matrixInt@{matrixInt}}
\index{matrixInt@{matrixInt}!matrixInt@{matrix\-Int}}
\paragraph{\setlength{\rightskip}{0pt plus 5cm}matrix\-Int::matrix\-Int (int {\em dimension1}, int {\em dimension2})}\hfill



Definition at line 10 of file matrix\-Int.C.\small\begin{verbatim}00011 {
00012    createAndInitialize(dimension1,dimension2);
00013 }
\end{verbatim}\normalsize 
\label{matrixInt_a1}
\index{matrixInt@{matrix\-Int}!matrixInt@{matrixInt}}
\index{matrixInt@{matrixInt}!matrixInt@{matrix\-Int}}
\paragraph{\setlength{\rightskip}{0pt plus 5cm}matrix\-Int::matrix\-Int (matrix\-Int \& {\em orig})}\hfill



Definition at line 17 of file matrix\-Int.C.\small\begin{verbatim}00018 {
00019    dim1 = orig.dim1;
00020    dim2 = orig.dim2;
00021    thematrix = orig.thematrix;  // do not copy the array, just the pointer...
00022 }
\end{verbatim}\normalsize 
\label{matrixInt_a2}
\index{matrixInt@{matrix\-Int}!~matrixInt@{$\sim$matrixInt}}
\index{~matrixInt@{$\sim$matrixInt}!matrixInt@{matrix\-Int}}
\paragraph{\setlength{\rightskip}{0pt plus 5cm}matrix\-Int::$\sim$matrix\-Int ()}\hfill



Definition at line 26 of file matrix\-Int.C.\small\begin{verbatim}00027 {
00028    for (int i=0; i<dim1; ++i)
00029       delete thematrix[i];
00030    delete thematrix; 
00031 }
\end{verbatim}\normalsize 


\subsubsection{Member Function Documentation}
\label{matrixInt_c0}
\index{matrixInt@{matrix\-Int}!createAndInitialize@{createAndInitialize}}
\index{createAndInitialize@{createAndInitialize}!matrixInt@{matrix\-Int}}
\paragraph{\setlength{\rightskip}{0pt plus 5cm}void matrix\-Int::create\-And\-Initialize (int {\em dimension1}, int {\em dimension2})\hspace{0.3cm}{\tt  [private]}}\hfill



Definition at line 35 of file matrix\-Int.C.

Referenced by {\bf matrix\-Int}().\small\begin{verbatim}00036 {
00037    dim1 = dimension1;
00038    dim2 = dimension2;
00039    thematrix = new (int *[dim1]);
00040    for (int i=0; i<dim1; ++i)
00041    {
00042       int *subarray = new int[dim2];
00043       for (int j=0; j<dim2; ++j)
00044      subarray[j] = 0;   // initialize the matrix to null.
00045       thematrix[i] = subarray;
00046    }
00047 }
\end{verbatim}\normalsize 
\label{matrixInt_a4}
\index{matrixInt@{matrix\-Int}!dimension1@{dimension1}}
\index{dimension1@{dimension1}!matrixInt@{matrix\-Int}}
\paragraph{\setlength{\rightskip}{0pt plus 5cm}int matrix\-Int::dimension1 ()\hspace{0.3cm}{\tt  [inline]}}\hfill



Definition at line 26 of file matrix\-Int.h.

Referenced by {\bf dist\-Info::numparts}(), and {\bf connect\-Info::numparts}().\small\begin{verbatim}00026 {return dim1;}
\end{verbatim}\normalsize 
\label{matrixInt_a5}
\index{matrixInt@{matrix\-Int}!dimension2@{dimension2}}
\index{dimension2@{dimension2}!matrixInt@{matrix\-Int}}
\paragraph{\setlength{\rightskip}{0pt plus 5cm}int matrix\-Int::dimension2 ()\hspace{0.3cm}{\tt  [inline]}}\hfill



Definition at line 27 of file matrix\-Int.h.\small\begin{verbatim}00027 {return dim2;}
\end{verbatim}\normalsize 
\label{matrixInt_a7}
\index{matrixInt@{matrix\-Int}!exchangeColumns@{exchangeColumns}}
\index{exchangeColumns@{exchangeColumns}!matrixInt@{matrix\-Int}}
\paragraph{\setlength{\rightskip}{0pt plus 5cm}matrix\-Int \& matrix\-Int::exchange\-Columns (int {\em a}, int {\em b})}\hfill



Definition at line 60 of file matrix\-Int.C.

Referenced by {\bf dist\-Info::exchange}().\small\begin{verbatim}00061 {
00062    int i = 0;
00063    while (i < dim1)
00064    {
00065       int *thisRow = thematrix[i++];
00066       int tmp    = thisRow[a];
00067       thisRow[a] = thisRow[b];
00068       thisRow[b] = tmp;
00069    }
00070    return *this;
00071 }
\end{verbatim}\normalsize 
\label{matrixInt_a6}
\index{matrixInt@{matrix\-Int}!exchangeRows@{exchangeRows}}
\index{exchangeRows@{exchangeRows}!matrixInt@{matrix\-Int}}
\paragraph{\setlength{\rightskip}{0pt plus 5cm}matrix\-Int \& matrix\-Int::exchange\-Rows (int {\em a}, int {\em b})}\hfill



Definition at line 51 of file matrix\-Int.C.

Referenced by {\bf dist\-Info::exchange}().\small\begin{verbatim}00052 {
00053    int *tmpRow  = thematrix[a];
00054    thematrix[a] = thematrix[b];
00055    thematrix[b] = tmpRow;
00056    return *this;
00057 }
\end{verbatim}\normalsize 
\label{matrixInt_a3}
\index{matrixInt@{matrix\-Int}!operator[]@{operator[]}}
\index{operator[]@{operator[]}!matrixInt@{matrix\-Int}}
\paragraph{\setlength{\rightskip}{0pt plus 5cm}int $\ast$ matrix\-Int::operator[$\,$] (int {\em i})\hspace{0.3cm}{\tt  [inline]}}\hfill



Definition at line 25 of file matrix\-Int.h.\small\begin{verbatim}00025 {return thematrix[i];}
\end{verbatim}\normalsize 
\label{matrixInt_a8}
\index{matrixInt@{matrix\-Int}!print@{print}}
\index{print@{print}!matrixInt@{matrix\-Int}}
\paragraph{\setlength{\rightskip}{0pt plus 5cm}void matrix\-Int::print ({\bf Transposition} {\em transp} = Not\-Transposed)}\hfill



Definition at line 75 of file matrix\-Int.C.\small\begin{verbatim}00076 {
00077    int vertical = transp==Transposed ? dim2 : dim1;
00078    int horizontal = transp==Transposed ? dim1 : dim2;
00079    for (int i=0; i<vertical; ++i)
00080    {
00081       cout << "\n";
00082       for (int j=0; j<horizontal; ++j)
00083      cout << form(" %3d", transp==Transposed ? thematrix[j][i] : thematrix[i][j])
00084           << flush;
00085    }
00086    cout << "\n\n" << flush;
00087 }
\end{verbatim}\normalsize 


\subsubsection{Member Data Documentation}
\label{matrixInt_o1}
\index{matrixInt@{matrix\-Int}!dim1@{dim1}}
\index{dim1@{dim1}!matrixInt@{matrix\-Int}}
\paragraph{\setlength{\rightskip}{0pt plus 5cm}int matrix\-Int::dim1\hspace{0.3cm}{\tt  [private]}}\hfill



Definition at line 19 of file matrix\-Int.h.\label{matrixInt_o2}
\index{matrixInt@{matrix\-Int}!dim2@{dim2}}
\index{dim2@{dim2}!matrixInt@{matrix\-Int}}
\paragraph{\setlength{\rightskip}{0pt plus 5cm}int matrix\-Int::dim2\hspace{0.3cm}{\tt  [private]}}\hfill



Definition at line 19 of file matrix\-Int.h.\label{matrixInt_o0}
\index{matrixInt@{matrix\-Int}!thematrix@{thematrix}}
\index{thematrix@{thematrix}!matrixInt@{matrix\-Int}}
\paragraph{\setlength{\rightskip}{0pt plus 5cm}int $\ast$$\ast$ matrix\-Int::thematrix\hspace{0.3cm}{\tt  [private]}}\hfill



Definition at line 18 of file matrix\-Int.h.

The documentation for this class was generated from the following files:\begin{CompactItemize}
\item 
{\bf matrix\-Int.h}\item 
{\bf matrix\-Int.C}\end{CompactItemize}
