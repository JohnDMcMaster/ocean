\subsection{pattern.C File Reference}
\label{pattern.C}\index{pattern.C@{pattern.C}}
{\tt \#include $<$stdlib.h$>$}\par
{\tt \#include $<$string.h$>$}\par
{\tt \#include $<$math.h$>$}\par
{\tt \#include "pattern.h"}\par
{\tt \#include "clst.h"}\par
\subsubsection*{Functions}
\begin{CompactItemize}
\item 
{\bf layer\-Type} {\bf recognize\-Layer} (short layer)
\end{CompactItemize}


\subsubsection{Function Documentation}
\label{pattern.C_a0}
\index{pattern.C@{pattern.C}!recognizeLayer@{recognizeLayer}}
\index{recognizeLayer@{recognizeLayer}!pattern.C@{pattern.C}}
\paragraph{\setlength{\rightskip}{0pt plus 5cm}{\bf layer\-Type} recognize\-Layer (short {\em layer})}\hfill



Definition at line 218 of file pattern.C.\small\begin{verbatim}00221 {
00222   layerType l;
00223 
00224   switch (layer)
00225     {
00226     case 100:
00227       l = ViaLayer;
00228       break;
00229     case 1:
00230       l = Metal1Layer;  
00231       break;
00232     case 2:
00233       l = Metal2Layer;  
00234       break;
00235     case 101:
00236       l = MetalsViaLayer;  
00237       break;
00238     default:
00239       l = Metal2Layer;
00240     }
00241   return l;
00242 
00243 }
\end{verbatim}\normalsize 
