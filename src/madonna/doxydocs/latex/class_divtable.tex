\subsection{Div\-Table  Class Reference}
\label{DivTable}\index{DivTable@{Div\-Table}}
{\tt \#include $<$div\-Table.h$>$}

Inheritance diagram for Div\-Table:\begin{figure}[H]
\begin{center}
\leavevmode
\setlength{\epsfysize}{6cm}
\epsfbox{class_divtable.eps}
\end{center}
\end{figure}
\subsubsection*{Public Methods}
\begin{CompactItemize}
\item 
{\bf Div\-Table} ( char $\ast$file\-Name = "div.tab", char $\ast$file\-Name2="seadif/div.tab" )
\item 
{\bf $\sim$Div\-Table} ()
\item 
virtual {\bf class\-Type} {\bf my\-No} () const
\item 
virtual char$\ast$ {\bf my\-Name} () const
\end{CompactItemize}


\subsubsection{Constructor \& Destructor Documentation}
\label{DivTable_a0}
\index{DivTable@{Div\-Table}!DivTable@{DivTable}}
\index{DivTable@{DivTable}!DivTable@{Div\-Table}}
\paragraph{\setlength{\rightskip}{0pt plus 5cm}Div\-Table::Div\-Table (char $\ast$ {\em file\-Name} = "div.tab", char $\ast$ {\em file\-Name2} = "seadif/div.tab")}\hfill



Definition at line 12 of file div\-Table.C.\small\begin{verbatim}00012 :Array(10,0,10)
00013 //
00014 // Constructor : reads our table from file.
00015 {
00016   int i,x,y;
00017   char c;
00018   ifstream ins;
00019 
00020   ins.open(fileName);
00021   if (!ins) ins.open(fileName2);
00022 
00023   if(!ins)
00024   {
00025     cerr << "\n WARNING : Cannot find file " << fileName << " or " << fileName2 << "\n";
00026     cerr << "   The default way of dividing a rectangle into n \n";
00027     cerr << "   parts  will be used .. " << endl << endl;
00028     
00029     Point *newDiv= new Point(2,2);
00030     addAt( *newDiv ,4);
00031 
00032     newDiv = new Point(3,3);
00033     addAt( *newDiv ,9);
00034 
00035     newDiv = new Point(4,4);
00036     addAt( *newDiv ,16);
00037 
00038     newDiv = new Point(5,4);
00039     addAt( *newDiv ,20);
00040 
00041     newDiv = new Point(6,4);
00042     addAt( *newDiv ,24);
00043 
00044     newDiv = new Point(8,4);
00045     addAt( *newDiv ,32);
00046 
00047     newDiv = new Point(16,4);
00048     addAt( *newDiv ,64);
00049 
00050     newDiv = new Point(16,8);
00051     addAt( *newDiv ,128);
00052 
00053     newDiv = new Point(32,8);
00054     addAt( *newDiv ,256);
00055 
00056     newDiv = new Point(64,8);
00057     addAt( *newDiv ,512);
00058 
00059 
00060   }
00061   while ( ins) 
00062   {
00063     ins >> i >> x >> y;
00064     if (ins.rdstate() & ios::eofbit)
00065       return;
00066 
00067     if (!ins)
00068       usrErr("DivTable::DivTable",EINPDAT);
00069     
00070     //  now skip everything to the end of the line
00071 
00072     while(ins.get(c) && c!='\n');
00073 
00074     Point &newDiv= *new Point(x,y);
00075     addAt( newDiv ,i);
00076   }
00077   
00078 }
\end{verbatim}\normalsize 
\label{DivTable_a1}
\index{DivTable@{Div\-Table}!~DivTable@{$\sim$DivTable}}
\index{~DivTable@{$\sim$DivTable}!DivTable@{Div\-Table}}
\paragraph{\setlength{\rightskip}{0pt plus 5cm}Div\-Table::$\sim$Div\-Table ()\hspace{0.3cm}{\tt  [inline]}}\hfill



Definition at line 27 of file div\-Table.h.\small\begin{verbatim}00027 {}
\end{verbatim}\normalsize 


\subsubsection{Member Function Documentation}
\label{DivTable_a3}
\index{DivTable@{Div\-Table}!myName@{myName}}
\index{myName@{myName}!DivTable@{Div\-Table}}
\paragraph{\setlength{\rightskip}{0pt plus 5cm}char $\ast$ Div\-Table::my\-Name () const\hspace{0.3cm}{\tt  [inline, virtual]}}\hfill



Reimplemented from {\bf Array} {\rm (p.\,\pageref{Array_a6})}.

Definition at line 30 of file div\-Table.h.\small\begin{verbatim}00030 { return "DivTab"; }
\end{verbatim}\normalsize 
\label{DivTable_a2}
\index{DivTable@{Div\-Table}!myNo@{myNo}}
\index{myNo@{myNo}!DivTable@{Div\-Table}}
\paragraph{\setlength{\rightskip}{0pt plus 5cm}{\bf class\-Type} Div\-Table::my\-No () const\hspace{0.3cm}{\tt  [inline, virtual]}}\hfill



Reimplemented from {\bf Array} {\rm (p.\,\pageref{Array_a5})}.

Definition at line 29 of file div\-Table.h.\small\begin{verbatim}00029 { return DivTabClass; }
\end{verbatim}\normalsize 


The documentation for this class was generated from the following files:\begin{CompactItemize}
\item 
{\bf div\-Table.h}\item 
{\bf div\-Table.C}\end{CompactItemize}
