\subsection{main.C File Reference}
\label{main.C}\index{main.C@{main.C}}
{\tt \#include $<$stdio.h$>$}\par
{\tt \#include $<$stream.h$>$}\par
{\tt \#include $<$stdlib.h$>$}\par
{\tt \#include $<$string.h$>$}\par
{\tt \#include $<$sys/types.h$>$}\par
{\tt \#include $<$sys/times.h$>$}\par
{\tt \#include $<$unistd.h$>$}\par
{\tt \#include $<$new.h$>$}\par
{\tt \#include $<$sea.h$>$}\par
{\tt \#include $<$genpart.h$>$}\par
{\tt \#include "sdferrors.h"}\par
\subsubsection*{Defines}
\begin{CompactItemize}
\item 
\#define {\bf PRECISION}\ 10
\item 
\#define {\bf INITIAL\_\-TEMPERATURE}\ 0.50
\item 
\#define {\bf INITIAL\_\-COOLING}\ 0.99
\item 
\#define {\bf NOTINITIALIZED}\ -1
\item 
\#define {\bf MAXSTR}\ 300
\item 
\#define {\bf SDFWRITEALLMASK}\ 0x8000
\item 
\#define {\bf MAXSTR}\ 300
\end{CompactItemize}
\subsubsection*{Functions}
\begin{CompactItemize}
\item 
CIRCUITPTR {\bf madonna} (CIRCUIT $\ast$)
\item 
void {\bf mark\-Childs} (CIRCUIT$\ast$)
\item 
void {\bf my\_\-new\_\-handler} (void)
\item 
void {\bf part\-Write\-All\-Cir} (long ,CIRCUITPTR )
\item 
void {\bf part\-Write\-All\-Cir\_\-1} (CIRCUITPTR )
\item 
void {\bf part\-Write\-All\-Cir\_\-2} (long,CIRCUITPTR )
\item 
int {\bf check\-That\-Circuit\-Looks\-Reasonable} (CIRCUITPTR)
\item 
void {\bf printstatisticsinstatusfield} (LAYOUT$\ast$,int,int)
\item 
int {\bf main} (int argc,char $\ast$argv[$\,$])
\item 
int {\bf costmincut} (int netdistr[$\,$],int numparts)
\item 
void {\bf printpartstat} ({\bf TOTALPPTR} total)
\item 
char$\ast$ {\bf bname} (char $\ast$s)
\end{CompactItemize}
\subsubsection*{Variables}
\begin{CompactItemize}
\item 
int {\bf requested\-Grid\-Points} [3] = \{NOTINITIALIZED, NOTINITIALIZED, NOTINITIALIZED\}
\item 
int {\bf expandable\-Direction} = NOTINITIALIZED
\item 
char$\ast$ {\bf thedate} = THEDATE
\item 
char $\ast$ {\bf thehost} = THEHOST
\item 
char$\ast$ {\bf madonna\_\-version} = "3.2"
\item 
int {\bf dont\-Check\-Child\-Ports}
\item 
int {\bf highnumpart} = 16
\item 
int {\bf stopquotient} = 100
\item 
int {\bf processid}
\item 
int {\bf madonnamakeminiplaza} = TRUE
\item 
int {\bf madonnamakepartition} = TRUE
\item 
int {\bf permutate\-Clusters\-At\-Exit} = TRUE
\item 
int {\bf max\-Number\-Of\-Cluster\-Permutations} = 2
\item 
int {\bf madonna\-Allow\-Random\-Moves} = NIL
\item 
int {\bf accept\-Candidate\-Even\-If\-Negative\-Gain} = NIL
\item 
int {\bf set\_\-srand} = 0
\item 
int {\bf phil\_\-verbose} = 0
\item 
int {\bf rand\_\-points} = 1
\item 
int {\bf do\-Compresion} = 1
\item 
double {\bf extraplaza} = 1.0
\item 
int {\bf macro\-Min\-Size} = 100
\item 
int {\bf slicing\-Layout} = 1
\item 
int {\bf do\-Tran\-Ana} = -1
\item 
int {\bf make\-Channels} = 0
\item 
char$\ast$ {\bf circuit\_\-name} = NIL
\item 
char $\ast$ {\bf function\_\-name} = NIL
\item 
char $\ast$ {\bf library\_\-name} = NIL
\item 
char $\ast$ {\bf layoutname} = NIL
\item 
STRING {\bf Route\-Ascii\-File} = NIL
\item 
double {\bf initial\_\-temperature} = INITIAL\_\-TEMPERATURE
\item 
double {\bf initial\_\-cooling} = INITIAL\_\-COOLING
\item 
filebuf {\bf Route\-Ascii\-Stream\-Buf}
\item 
char {\bf strng} [1+MAXSTR]
\end{CompactItemize}


\subsubsection{Define Documentation}
\label{main.C_a2}
\index{main.C@{main.C}!INITIAL_COOLING@{INITIAL\_\-COOLING}}
\index{INITIAL_COOLING@{INITIAL\_\-COOLING}!main.C@{main.C}}
\paragraph{\setlength{\rightskip}{0pt plus 5cm}\#define INITIAL\_\-COOLING\ 0.99}\hfill



Definition at line 26 of file main.C.\label{main.C_a1}
\index{main.C@{main.C}!INITIAL_TEMPERATURE@{INITIAL\_\-TEMPERATURE}}
\index{INITIAL_TEMPERATURE@{INITIAL\_\-TEMPERATURE}!main.C@{main.C}}
\paragraph{\setlength{\rightskip}{0pt plus 5cm}\#define INITIAL\_\-TEMPERATURE\ 0.50}\hfill



Definition at line 25 of file main.C.\label{main.C_a6}
\index{main.C@{main.C}!MAXSTR@{MAXSTR}}
\index{MAXSTR@{MAXSTR}!main.C@{main.C}}
\paragraph{\setlength{\rightskip}{0pt plus 5cm}\#define MAXSTR\ 300}\hfill



Definition at line 615 of file main.C.\label{main.C_a4}
\index{main.C@{main.C}!MAXSTR@{MAXSTR}}
\index{MAXSTR@{MAXSTR}!main.C@{main.C}}
\paragraph{\setlength{\rightskip}{0pt plus 5cm}\#define MAXSTR\ 300}\hfill



Definition at line 615 of file main.C.\label{main.C_a3}
\index{main.C@{main.C}!NOTINITIALIZED@{NOTINITIALIZED}}
\index{NOTINITIALIZED@{NOTINITIALIZED}!main.C@{main.C}}
\paragraph{\setlength{\rightskip}{0pt plus 5cm}\#define NOTINITIALIZED\ -1}\hfill



Definition at line 43 of file main.C.\label{main.C_a0}
\index{main.C@{main.C}!PRECISION@{PRECISION}}
\index{PRECISION@{PRECISION}!main.C@{main.C}}
\paragraph{\setlength{\rightskip}{0pt plus 5cm}\#define PRECISION\ 10}\hfill



Definition at line 24 of file main.C.\label{main.C_a5}
\index{main.C@{main.C}!SDFWRITEALLMASK@{SDFWRITEALLMASK}}
\index{SDFWRITEALLMASK@{SDFWRITEALLMASK}!main.C@{main.C}}
\paragraph{\setlength{\rightskip}{0pt plus 5cm}\#define SDFWRITEALLMASK\ 0x8000}\hfill



Definition at line 573 of file main.C.

\subsubsection{Function Documentation}
\label{main.C_a18}
\index{main.C@{main.C}!bname@{bname}}
\index{bname@{bname}!main.C@{main.C}}
\paragraph{\setlength{\rightskip}{0pt plus 5cm}char $\ast$ bname (char $\ast$ {\em s})\hspace{0.3cm}{\tt  [static]}}\hfill



Definition at line 670 of file main.C.

Referenced by {\bf main}().\small\begin{verbatim}00671 {
00672 char *p = strrchr(s,'/');
00673 if (p)
00674    return p+1;
00675 else
00676    return s;
00677 }
\end{verbatim}\normalsize 
\label{main.C_a13}
\index{main.C@{main.C}!checkThatCircuitLooksReasonable@{checkThatCircuitLooksReasonable}}
\index{checkThatCircuitLooksReasonable@{checkThatCircuitLooksReasonable}!main.C@{main.C}}
\paragraph{\setlength{\rightskip}{0pt plus 5cm}int check\-That\-Circuit\-Looks\-Reasonable (CIRCUITPTR {\em thecircuit})\hspace{0.3cm}{\tt  [static]}}\hfill



Definition at line 683 of file main.C.

Referenced by {\bf main}().\small\begin{verbatim}00684 {
00685    int numInst = 0;
00686    int numTrans = 0;
00687    STRING nenhStr = cs("nenh");
00688    STRING penhStr = cs("penh");
00689    STRING capStr = cs("cap");
00690    STRING resStr = cs("res");
00691 
00692    for (CIRINSTPTR cinst=thecircuit->cirinst; cinst!=NIL; cinst=cinst->next)
00693    {
00694       if (cinst->circuit->name == nenhStr || cinst->circuit->name == penhStr ||
00695       cinst->circuit->name == capStr || cinst->circuit->name == resStr)
00696      numTrans += 1;
00697       numInst += 1;
00698    }
00699    fs(nenhStr);
00700    fs(penhStr);
00701    fs(capStr);
00702    fs(resStr);
00703    if (double(numTrans)/numInst > 0.5)
00704       return NIL;       // too much transistors...
00705    return TRUE;  
00706 }
\end{verbatim}\normalsize 
\label{main.C_a16}
\index{main.C@{main.C}!costmincut@{costmincut}}
\index{costmincut@{costmincut}!main.C@{main.C}}
\paragraph{\setlength{\rightskip}{0pt plus 5cm}int costmincut (int {\em netdistr}[$\,$], int {\em numparts})}\hfill



Definition at line 643 of file main.C.\small\begin{verbatim}00644 {
00645 int cost=0;
00646 
00647 if (numparts!=2)
00648    {
00649    fprintf(stderr,"costmincut: cannot compute cost for %1d partitions!\n",numparts);
00650    dumpcore();
00651    }
00652 if (netdistr[0]>0 && netdistr[1]>0)
00653    cost=1;
00654 return(cost);
00655 }
\end{verbatim}\normalsize 
\label{main.C_a7}
\index{main.C@{main.C}!madonna@{madonna}}
\index{madonna@{madonna}!main.C@{main.C}}
\paragraph{\setlength{\rightskip}{0pt plus 5cm}CIRCUITPTR madonna (CIRCUIT $\ast$ {\em circuit})}\hfill



Definition at line 76 of file madonna.C.

Referenced by {\bf main}().\small\begin{verbatim}00077 {
00078    TOTALPPTR    total=NIL;
00079    FUNCTIONPTR  savethisfun=thisfun;
00080    CIRCUITPTR   savethiscir=thiscir,thecircuit,bestcircuit;
00081    LAYOUTPTR    savethislay=thislay;
00082    LIBRARYPTR   savethislib=thislib;
00083    
00084    initnetcostinfo();
00085 
00086    readImageFile();         // Here we call the parser to read all 
00087                                     // the stuff about image
00088    
00089    printNetlistStatistics(circuit);
00090 
00091    if (madonnamakepartition)
00092    {
00093       madonna_(&total,circuit,1);
00094 
00095 #ifdef __MSDOS__
00096       cerr << "\nAvailable memory: " << coreleft() << "\n\n";
00097 #endif
00098 
00099       bestcircuit=(total ? total->bestpart : NIL);
00100    }
00101    else
00102       bestcircuit=NIL;
00103 
00104    if (total==NIL || total->bestpart==NIL)
00105       thecircuit=circuit;
00106    else
00107       thecircuit=total->bestpart;
00108    if (madonnamakeminiplaza )
00109    {
00110 
00111       if(! madonnamakepartition )
00112       {
00113      phil(circuit_name,function_name,library_name,
00114           layoutname,extraplaza,NULL,NULL,NULL);
00115       }
00116       else
00117       {
00118      if (total != NIL && total->nx >= 1 && total->ny >= 1)
00119      {
00120         // ...and now for the global routing:
00121         expansionGrid globgrid(total); // create global grid and globl nets
00122         globgrid.routeGlobNets();      // route the nets
00123 
00124         phil(circuit_name,function_name,library_name,layoutname,
00125          extraplaza,thecircuit,savethiscir,total->routing);
00126      }
00127      else
00128      {
00129         // nothing to do for the global router ...
00130         phil(circuit_name,function_name,library_name,layoutname,
00131          extraplaza,thecircuit,savethiscir,NIL);
00132      }
00133       }
00134 #ifdef __MSDOS__
00135       cerr << "\nAvailable memory: " << coreleft() << "\n\n";
00136 #endif
00137    }
00138    thislay=savethislay; thiscir=savethiscir;
00139    thisfun=savethisfun; thislib=savethislib;
00140    
00141    return bestcircuit;
00142 }
\end{verbatim}\normalsize 
\label{main.C_a15}
\index{main.C@{main.C}!main@{main}}
\index{main@{main}!main.C@{main.C}}
\paragraph{\setlength{\rightskip}{0pt plus 5cm}int main (int {\em argc}, char $\ast$ {\em argv}[$\,$])}\hfill



Definition at line 114 of file main.C.\small\begin{verbatim}00115 {
00116    
00117    extern int  optind;
00118    extern char *optarg;
00119    int         i,readonlythisthing=NIL,doflattenfirst=NIL,
00120    doflattenonly=NIL,checkNetlistFirst=TRUE;
00121    
00122    CIRCUITPTR  madonnapartitionedcircuit;
00123    
00124    CIRCUITPTR  thecircuit;
00125    
00126    
00127 #ifdef __MSDOS__
00128    
00129    fnsplit(argv[0],drive,dir,file,ext);
00130    
00131 #endif
00132    
00133    setvbuf(stdout,NULL,_IOLBF,NULL);
00134    setvbuf(stderr,NULL,_IOLBF,NULL);
00135    printf("\nMadonna version %s (compiled %s on \"%s\")\n\n",
00136       madonna_version,thedate,thehost);
00137 
00138    set_new_handler(&my_new_handler); // type message if operator new() fails
00139 
00140    /* Parse options */
00141    while ((i=getopt(argc,argv,
00142             "l:f:c:S:haAbrtT:viuj:s:d:q:o:mMpP:kLe:x:y:gwW:n:R:C")) != EOF)
00143    {
00144       switch (i)
00145       {
00146       case 'h':   /* print help */
00147      fprintf(stdout,"usage: %s [..options..] [circuit_name]\n",
00148 #ifdef __MSDOS__
00149          file
00150 #else
00151          argv[0]
00152 #endif
00153          );
00154      printf
00155         ("options:\n"
00156          "-x<size> preferred width of the placement in gridpoints\n"
00157          "-y<size> preferred heigth of the placement in gridpoints\n"
00158          "-e<dir>  where <dir> is either 'x' or 'y': direction to grow\n"
00159          "-c<name> the circuit name\n"
00160          "-f<name> the function name\n"
00161          "-l<name> the library name\n"
00162          "-o<layoutname> name of output (default Madonna/Prince)\n"
00163          "-p do not exhaustively permutate the partitions at exit\n"
00164          "-P<num> permutate at most <num> times the partitions\n"
00165          "-S<size> set minimal \"huge\" cell size, default 100 b.c.\n"
00166          "-M never treat cells as \"huge\" \n"
00167          "-q set stop quotient (default 100)\n"
00168          "-v debugging mode for detailed placement\n"
00169          "-a not random point selection in detailed placement\n"
00170          "-b do not do compaction after detailed placement\n"
00171          "-j<number> positive value to initialize random generator\n"
00172          "-g do not check that at most 50%% of the instances are transistors\n"
00173          "-R<fname> set file name for ascii dump of global routes\n"
00174          "-L create flat layout instead of slicing structure\n" 
00175          "-T<num> num==0 don't do transparency analysis,num==1 do it.\n"  
00176          "-C allocate place for channels before placement (default off).\n"
00177          "-w allow random moves\n"
00178          "-W<temp> start with <temp> percent random moves (def=%1.2f)\n"
00179          "-d<numparts> don't partition beyond numparts\n"
00180          "-r only read circuit, then exit\n"
00181          "-m do not make miniplaza\n"
00182          "-n<cool> multiply <temp> by <cool> after each random move (def=%1.2f)\n"
00183          "-s<size> set initial plaza magnification, default is 2.0\n"
00184          "-t flatten before partitioning\n"
00185          "-u only  flatten circuit, no partitioning.\n"
00186          "-k do not lock the Seadif database while running madonna\n",
00187          (float)initial_temperature,(float)initial_cooling);
00188      exit(0);
00189      break;
00190       case 'c':   /* circuit name */
00191      circuit_name = cs(optarg);
00192      break;
00193       case 'f':   /* function name */
00194      function_name = cs(optarg);
00195      break;
00196       case 'j':   /* random gen seed*/
00197      set_srand = atoi(optarg);
00198      break;
00199       case 'a':   /* detailed placement random or sequential */
00200      rand_points = 0;
00201      break;
00202       case 'b':   /* do not do compaction after placement */
00203      doCompresion = 0;
00204      break;
00205       case 'l':   /* library name */
00206      library_name = cs(optarg);
00207      break;
00208       case 'k':
00209          {extern int sdfmakelockfiles; sdfmakelockfiles=NIL;}
00210      break;
00211       case 'r':
00212      readonlythisthing=TRUE;
00213      break;
00214       case 'R': // set file name for ascii dump of global routes
00215      RouteAsciiFile = cs(optarg);
00216      break;
00217       case 'm':
00218      madonnamakeminiplaza=NIL;
00219      break;
00220       case 'p':
00221      permutateClustersAtExit = NIL;
00222      break;
00223       case 'P':
00224      maxNumberOfClusterPermutations = atoi(optarg);
00225      break;
00226       case 'L':
00227      slicingLayout=0;
00228      break;
00229       case 'T':
00230      doTranAna = atoi(optarg);
00231      break;
00232       case 'C':
00233          makeChannels=1;
00234      break;
00235       case 't':
00236      doflattenfirst=TRUE;
00237      break;
00238       case 'v':
00239      phil_verbose=TRUE;
00240      break;
00241       case 'u':
00242      doflattenonly=TRUE;
00243      break;
00244       case 's':
00245      extraplaza=atof(optarg);
00246      break;
00247       case 'S':
00248      macroMinSize=atoi(optarg);
00249      break;
00250       case 'M':
00251      macroMinSize=MAXINT;
00252      break;
00253       case 'q':
00254      stopquotient=atoi(optarg);
00255      break;
00256       case 'd':
00257      highnumpart=atoi(optarg);
00258      break;
00259       case 'o':
00260      layoutname = cs(optarg);
00261      break;
00262       case 'g':
00263      checkNetlistFirst = NIL;
00264      break;
00265       case 'x':   /* requestedGridPoints[HOR]; */
00266          if ((requestedGridPoints[HOR] = atoi(optarg)) <= 0)
00267      {
00268         printf("option '-x' requires a positive integer\n");
00269         exit(1);
00270      }
00271          break;
00272       case 'y':   /* requestedGridPoints[VER]; */
00273          if ((requestedGridPoints[VER] = atoi(optarg)) <= 0)
00274      {
00275         printf("option '-y' requires a positive integer\n");
00276         exit(1);
00277      }
00278          break;
00279       case 'e':   /* expandable direction */
00280      switch (optarg[0])
00281      {
00282      case 'x':
00283      case 'X':
00284         expandableDirection = HOR;
00285         break;
00286      case 'y':
00287      case 'Y':
00288         expandableDirection = VER;
00289         break;
00290      default:
00291         cout << "illegal argument to -e option, must be 'x' or 'y'...\n";
00292         exit(1);
00293         break;
00294      }
00295      break;
00296       case 'w': /* allow random moves */
00297      madonnaAllowRandomMoves = TRUE;
00298      break;
00299       case 'W': /* starting temperature, must be in range [0..1] */
00300      madonnaAllowRandomMoves = TRUE;
00301      initial_temperature = atof(optarg);
00302      if (initial_temperature < 0.0 || initial_temperature > 1.0)
00303      {
00304         fprintf(stderr,"initial temperature must be in range [0..1]\n");
00305         exit(1);
00306      }
00307      break;
00308       case 'n': /* cooling speed, must be in range [0..1] */
00309      madonnaAllowRandomMoves = TRUE;
00310      initial_cooling = atof(optarg);
00311      if (initial_cooling < 0.0 || initial_cooling > 1.0)
00312      {
00313         fprintf(stderr,"initial cooling must be in range [0..1]\n");
00314         exit(1);
00315      }
00316      break;
00317       case 'A':
00318      acceptCandidateEvenIfNegativeGain = TRUE;
00319      break;
00320       case '?':
00321      fprintf(stderr,"\nIllegal argument.\n\n");
00322      exit(1);
00323      break;
00324       default:
00325      break;
00326       }
00327    }
00328    
00329    initsignals();
00330 
00331    if (expandableDirection == NOTINITIALIZED)
00332    {
00333       // grow vertically by default...:
00334       expandableDirection = VER;
00335    }
00336 
00337   
00338 #if 0
00339    // only compact the layout if did not ask for a box of a certain size:
00340    if (requestedGridPoints[HOR] != NOTINITIALIZED &&
00341        requestedGridPoints[VER] != NOTINITIALIZED )
00342       doCompresion = 0;
00343 #endif
00344 
00345    if(doTranAna == 1)
00346    {
00347      slicingLayout=1;
00348      rand_points=0;
00349      makeChannels=0;
00350    }
00351 
00352    if(makeChannels)
00353      doTranAna=0;
00354 
00355    if (circuit_name == NIL)
00356       if (optind == argc - 1)
00357      circuit_name = cs(argv[optind]);
00358       else
00359       {
00360      cout << "please specify a circuit name ...\n";
00361      exit(1);
00362       }
00363 
00364    dontCheckChildPorts = 1;
00365 
00366    if (function_name == NIL)
00367       function_name = cs(circuit_name);
00368 
00369    if (library_name == NIL)
00370    {
00371       if (sdfgetcwd()==NIL)
00372       {
00373      cout << "I cannot figure what's the current working directory...\n";
00374      exit(1);
00375       }
00376       library_name = cs(bname(sdfgetcwd()));
00377    }
00378 
00379    if ((i = sdfopen()) != SDF_NOERROR)
00380    {
00381       if (i == SDFERROR_FILELOCK)
00382       {
00383      cerr <<
00384         "ERROR: The seadif database is locked by another program.\n"
00385         "       Try again later, because only one program at the time\n"
00386         "       can access it. If you are sure that nobody else is\n"
00387         "       working on the database, you can remove the lockfiles.\n";
00388       }
00389       else
00390      cerr << "ERROR: cannot open seadif database.\n";
00391       sdfexit(i);
00392    }
00393 
00394    if (RouteAsciiFile != NIL)
00395       if (RouteAsciiStreamBuf.open(RouteAsciiFile, output) == 0)
00396       {
00397      cerr << "WARNING: cannot open file \"" << RouteAsciiFile
00398           << "\" to dump global routes on. (ignored)\n\n" << flush;
00399      RouteAsciiFile = NIL;
00400       }
00401 
00402    fprintf(stdout,"reading circuit \"%s(%s(%s))\" ...",
00403        circuit_name,function_name,library_name);
00404    fflush(stdout);
00405    if(sdfreadallcir(SDFCIRNETLIST+SDFCIRSTAT, circuit_name, function_name, library_name)==NIL)
00406    {
00407       fprintf(stdout,"\nERROR: apparently your cell \"%s(%s(%s))\" is out to lunch...\n",
00408           circuit_name, function_name, library_name);
00409       sdfexit(1);
00410    }
00411    fprintf(stdout,"done\n");
00412    if (thiscir->cirinst==NIL)
00413    {
00414       fprintf(stdout,"\nERROR: no instances found...\n");
00415       sdfexit(1);
00416    }
00417    if (readonlythisthing)
00418       sdfexit(0);
00419    thecircuit=thiscir;
00420    
00421    if (madonnamakepartition)
00422       markChilds(thiscir);
00423    
00424    if (doflattenfirst || doflattenonly)
00425       sdfflatcir(thecircuit);
00426    if (doflattenonly)
00427    {
00428       /* append "_f" to circuit name */
00429       char tmpstr[MAXSTR+1];
00430       strncpy(tmpstr,thecircuit->name,MAXSTR); strncat(tmpstr,"_f",MAXSTR);
00431       fs(thecircuit->name); thecircuit->name=cs(tmpstr);
00432       sdfwritecir(SDFCIRALL,thecircuit);
00433       sdfclose();
00434       sdfexit(0);
00435    }
00436 
00437    if (checkNetlistFirst)
00438       if (checkThatCircuitLooksReasonable(thecircuit) == NIL)
00439       {
00440      fprintf(stdout,
00441          "ERROR: your netlist contains too much transistors and/or\n"
00442          "       capacitors to my taste (> 50%%). I assume it is not\n"
00443          "       your intention to place this circuit. Use -g option\n"
00444          "       to enforce placement of this strange circuit...\n");
00445      sdfexit(SDFERROR_MADONNA_FAILED);
00446       }
00447 
00448    time_t     totaltime1,totaltime2;
00449    int        seconds;
00450    long       clock_ticks_per_second=100; /* 100 on hp9k835 */
00451 #ifndef __MSDOS__
00452    struct tms tmsbuf;
00453 #else
00454    time_t tms;
00455 #endif
00456    
00457 #ifndef __MSDOS__
00458    times(&tmsbuf);
00459    totaltime1=tmsbuf.tms_utime+tmsbuf.tms_stime+tmsbuf.tms_cutime+tmsbuf.tms_cstime;
00460 #else
00461    tms=time(NULL);
00462 #endif
00463    
00464    fprintf(stdout,"\nKissing madonna hello...\n");
00465    madonnapartitionedcircuit=madonna(thecircuit);
00466    fprintf(stdout,"Kissing madonna good bye...\n");
00467 #ifndef __MSDOS__
00468    times(&tmsbuf);
00469    totaltime2=tmsbuf.tms_utime+tmsbuf.tms_stime+tmsbuf.tms_cutime+tmsbuf.tms_cstime;
00470    
00471    /* clock_ticks_per_second=sysconf(_SC_CLK_TCK);*/
00472    seconds=int((PRECISION*(totaltime2-totaltime1))/clock_ticks_per_second);
00473    fprintf(stdout,"\nMadonna took %1d.%1d seconds of your cpu\n",
00474        seconds/PRECISION,seconds%PRECISION);
00475    if (madonnamakeminiplaza)
00476       printstatisticsinstatusfield(thecircuit->layout,seconds/PRECISION,highnumpart);
00477    
00478 #else
00479    seconds=time(NULL)-tms;
00480    fprintf(stdout,"\nMadonna took %d seconds of your cpu\n",
00481        seconds);
00482    if (madonnamakeminiplaza)
00483       printstatisticsinstatusfield(thecircuit->layout,seconds,highnumpart);
00484    
00485 #endif
00486    
00487    if (madonnamakepartition && !madonnamakeminiplaza && madonnapartitionedcircuit!=NIL)
00488    {
00489       fprintf(stdout,"writing circuit \"%s(%s(%s))\"\n",
00490           madonnapartitionedcircuit->name,madonnapartitionedcircuit->function->name,
00491           madonnapartitionedcircuit->function->library->name);
00492       partWriteAllCir(SDFCIRALL,madonnapartitionedcircuit);
00493    }
00494    
00495    
00496    /*
00497       
00498       if (thecircuit->layout!=NIL)
00499       {
00500      if (layoutname!=NIL)
00501      {
00502         fs(thecircuit->layout->name);
00503         thecircuit->layout->name=cs(layoutname);
00504      }
00505      fprintf(stdout,"writing layout \"%s(%s(%s(%s)))\"\n",
00506          thecircuit->layout->name,thecircuit->name,thecircuit->function->name,
00507          thecircuit->function->library->name);
00508      sdfwritealllay(SDFLAYALL,thecircuit->layout);
00509       }
00510    */
00511    
00512    
00513    sdfclose();
00514 
00515    if (RouteAsciiFile != NIL)
00516       RouteAsciiStreamBuf.close();
00517      
00518    exit(SDF_NOERROR);
00519    return(SDF_NOERROR);
00520 }
\end{verbatim}\normalsize 
\label{main.C_a8}
\index{main.C@{main.C}!markChilds@{markChilds}}
\index{markChilds@{markChilds}!main.C@{main.C}}
\paragraph{\setlength{\rightskip}{0pt plus 5cm}void mark\-Childs (CIRCUIT $\ast$ {\em c\-Ptr})}\hfill



Definition at line 523 of file main.C.

Referenced by {\bf main}().\small\begin{verbatim}00527 {
00528   char *keyName=",mad_prim";
00529   STATUS *stat;
00530 
00531   for(CIRINST *ciPtr=cPtr->cirinst;ciPtr != NULL;ciPtr=ciPtr->next)
00532   {
00533     CIRCUIT *cur = ciPtr->circuit;
00534 
00535     if (cur == NULL)
00536     {
00537       fprintf(stderr,
00538         "\n ill formed circuit instance %s \n, quitting ... \n\n",
00539         ciPtr->name);
00540       sdfexit(1);
00541     }
00542     if (cur->status == NULL)
00543     {
00544       NewStatus(stat);
00545       cur->status=stat;
00546       stat->timestamp=0;
00547       stat->author=cs("Madonna");
00548       stat->program=cs("");
00549     }
00550     if ( strstr(cur->status->program,keyName) ==NULL )
00551       // we have to append it
00552     {
00553       char buf[200];
00554 
00555       buf[0]='\0';
00556       strncat(buf,cur->status->program,200);
00557       strncat(buf,keyName,200);
00558       fs(cur->status->program);
00559       cur->status->program=cs(buf);
00560     }
00561 
00562   }
00563 }
\end{verbatim}\normalsize 
\label{main.C_a9}
\index{main.C@{main.C}!my_new_handler@{my\_\-new\_\-handler}}
\index{my_new_handler@{my\_\-new\_\-handler}!main.C@{main.C}}
\paragraph{\setlength{\rightskip}{0pt plus 5cm}void my\_\-new\_\-handler (void)\hspace{0.3cm}{\tt  [static]}}\hfill



Definition at line 710 of file main.C.\small\begin{verbatim}00711 {
00712    cerr << "\n"
00713     << "FATAL: I cannot allocate enough memory." << endl
00714     << "       Ask your sysop to configure more swap space ..." << endl;
00715    sdfexit(1);
00716 }
\end{verbatim}\normalsize 
\label{main.C_a10}
\index{main.C@{main.C}!partWriteAllCir@{partWriteAllCir}}
\index{partWriteAllCir@{partWriteAllCir}!main.C@{main.C}}
\paragraph{\setlength{\rightskip}{0pt plus 5cm}void part\-Write\-All\-Cir (long {\em what}, CIRCUITPTR {\em cir})\hspace{0.3cm}{\tt  [static]}}\hfill



Definition at line 575 of file main.C.

Referenced by {\bf main}().\small\begin{verbatim}00576 {
00577 partWriteAllCir_1(cir);      /* initialize flag bits */
00578 partWriteAllCir_2(what,cir);   /* perform the write */
00579 }
\end{verbatim}\normalsize 
\label{main.C_a11}
\index{main.C@{main.C}!partWriteAllCir_1@{partWriteAllCir\_\-1}}
\index{partWriteAllCir_1@{partWriteAllCir\_\-1}!main.C@{main.C}}
\paragraph{\setlength{\rightskip}{0pt plus 5cm}void part\-Write\-All\-Cir\_\-1 (CIRCUITPTR {\em cir})\hspace{0.3cm}{\tt  [static]}}\hfill



Definition at line 582 of file main.C.

Referenced by {\bf part\-Write\-All\-Cir}().\small\begin{verbatim}00583 {
00584 CIRINSTPTR  ci;
00585 
00586 if (cir->status!=NIL && issubstring(cir->status->program,"mad_prim"))
00587    cir->flag.l |= SDFWRITEALLMASK; /* do not write it to database */
00588 else
00589    {
00590    cir->flag.l &= ~SDFWRITEALLMASK; /* clear bit 'written' */
00591    for (ci=cir->cirinst; ci!=0; ci=ci->next)
00592       partWriteAllCir_1(ci->circuit);
00593    }
00594 }
\end{verbatim}\normalsize 
\label{main.C_a12}
\index{main.C@{main.C}!partWriteAllCir_2@{partWriteAllCir\_\-2}}
\index{partWriteAllCir_2@{partWriteAllCir\_\-2}!main.C@{main.C}}
\paragraph{\setlength{\rightskip}{0pt plus 5cm}void part\-Write\-All\-Cir\_\-2 (long {\em what}, CIRCUITPTR {\em cir})\hspace{0.3cm}{\tt  [static]}}\hfill



Definition at line 597 of file main.C.

Referenced by {\bf part\-Write\-All\-Cir}().\small\begin{verbatim}00598 {
00599 CIRINSTPTR ci;
00600 
00601 if (cir->flag.l & SDFWRITEALLMASK)
00602    return;        /* mad_prim or already wrote this one */
00603 what &= SDFCIRALL;
00604 if (!sdfwritecir(what,cir))
00605    err(7,"sdfwriteallcir_2: cannot write circuit");
00606 cir->flag.l |= SDFWRITEALLMASK;   /* mark as 'written' */
00607 for (ci=cir->cirinst; ci!=0; ci=ci->next)
00608    partWriteAllCir_2(what,ci->circuit);
00609 }
\end{verbatim}\normalsize 
\label{main.C_a17}
\index{main.C@{main.C}!printpartstat@{printpartstat}}
\index{printpartstat@{printpartstat}!main.C@{main.C}}
\paragraph{\setlength{\rightskip}{0pt plus 5cm}void printpartstat ({\bf TOTALPPTR} {\em total})}\hfill



Definition at line 659 of file main.C.\small\begin{verbatim}00660 {
00661 fprintf(stdout,"\n(genpart (%s(%s(%s)))",total->topcell->name,
00662   total->topcell->function->name,total->topcell->function->library->name);
00663 fprintf(stdout,"\n    (numparts %1d) (strtnetcost %1d) (bestnetcost %1d)",
00664   total->numparts,total->strtnetcost,total->bestnetcost);
00665 fprintf(stdout,"\n    (nmoves %1d) (area %1d)",total->nmoves,total->area);
00666 fprintf(stdout,")\n\n");
00667 }
\end{verbatim}\normalsize 
\label{main.C_a14}
\index{main.C@{main.C}!printstatisticsinstatusfield@{printstatisticsinstatusfield}}
\index{printstatisticsinstatusfield@{printstatisticsinstatusfield}!main.C@{main.C}}
\paragraph{\setlength{\rightskip}{0pt plus 5cm}void printstatisticsinstatusfield (LAYOUT $\ast$ {\em layout}, int {\em cputime}, int {\em numparts})\hspace{0.3cm}{\tt  [static]}}\hfill



Definition at line 619 of file main.C.

Referenced by {\bf main}().\small\begin{verbatim}00621 {
00622 STATUSPTR  status;
00623 
00624 if (layout==NIL)
00625    return;
00626 status=layout->status;
00627 if (status==NIL)
00628    {
00629    NewStatus(status);
00630    layout->status=status;
00631    }
00632 sprintf(strng,"cputime=%1ds,numparts=%1d,pid=%1d",cputime,numparts,processid);
00633 if (status->program!=NIL)
00634    { /* already some status information. don't destroy. */
00635    strncat(strng,"; ",MAXSTR);
00636    strncat(strng,status->program,MAXSTR);
00637    fs(status->program);
00638    }
00639 status->program=cs(strng);
00640 }
\end{verbatim}\normalsize 


\subsubsection{Variable Documentation}
\label{main.C_a47}
\index{main.C@{main.C}!RouteAsciiFile@{RouteAsciiFile}}
\index{RouteAsciiFile@{RouteAsciiFile}!main.C@{main.C}}
\paragraph{\setlength{\rightskip}{0pt plus 5cm}STRING Route\-Ascii\-File = NIL}\hfill



Definition at line 103 of file main.C.\label{main.C_a50}
\index{main.C@{main.C}!RouteAsciiStreamBuf@{RouteAsciiStreamBuf}}
\index{RouteAsciiStreamBuf@{RouteAsciiStreamBuf}!main.C@{main.C}}
\paragraph{\setlength{\rightskip}{0pt plus 5cm}filebuf Route\-Ascii\-Stream\-Buf}\hfill



Definition at line 108 of file main.C.\label{main.C_a33}
\index{main.C@{main.C}!acceptCandidateEvenIfNegativeGain@{acceptCandidateEvenIfNegativeGain}}
\index{acceptCandidateEvenIfNegativeGain@{acceptCandidateEvenIfNegativeGain}!main.C@{main.C}}
\paragraph{\setlength{\rightskip}{0pt plus 5cm}int accept\-Candidate\-Even\-If\-Negative\-Gain = NIL}\hfill



Definition at line 86 of file main.C.\label{main.C_a43}
\index{main.C@{main.C}!circuit_name@{circuit\_\-name}}
\index{circuit_name@{circuit\_\-name}!main.C@{main.C}}
\paragraph{\setlength{\rightskip}{0pt plus 5cm}char $\ast$ circuit\_\-name = NIL}\hfill



Definition at line 98 of file main.C.\label{main.C_a37}
\index{main.C@{main.C}!doCompresion@{doCompresion}}
\index{doCompresion@{doCompresion}!main.C@{main.C}}
\paragraph{\setlength{\rightskip}{0pt plus 5cm}int do\-Compresion = 1}\hfill



Definition at line 90 of file main.C.\label{main.C_a41}
\index{main.C@{main.C}!doTranAna@{doTranAna}}
\index{doTranAna@{doTranAna}!main.C@{main.C}}
\paragraph{\setlength{\rightskip}{0pt plus 5cm}int do\-Tran\-Ana = -1}\hfill



Definition at line 94 of file main.C.\label{main.C_a24}
\index{main.C@{main.C}!dontCheckChildPorts@{dontCheckChildPorts}}
\index{dontCheckChildPorts@{dontCheckChildPorts}!main.C@{main.C}}
\paragraph{\setlength{\rightskip}{0pt plus 5cm}int dont\-Check\-Child\-Ports}\hfill



Definition at line 72 of file main.C.\label{main.C_a20}
\index{main.C@{main.C}!expandableDirection@{expandableDirection}}
\index{expandableDirection@{expandableDirection}!main.C@{main.C}}
\paragraph{\setlength{\rightskip}{0pt plus 5cm}int expandable\-Direction = NOTINITIALIZED}\hfill



Definition at line 46 of file main.C.\label{main.C_a38}
\index{main.C@{main.C}!extraplaza@{extraplaza}}
\index{extraplaza@{extraplaza}!main.C@{main.C}}
\paragraph{\setlength{\rightskip}{0pt plus 5cm}double extraplaza = 1.0}\hfill



Definition at line 91 of file main.C.\label{main.C_a44}
\index{main.C@{main.C}!function_name@{function\_\-name}}
\index{function_name@{function\_\-name}!main.C@{main.C}}
\paragraph{\setlength{\rightskip}{0pt plus 5cm}char $\ast$ function\_\-name = NIL}\hfill



Definition at line 99 of file main.C.\label{main.C_a25}
\index{main.C@{main.C}!highnumpart@{highnumpart}}
\index{highnumpart@{highnumpart}!main.C@{main.C}}
\paragraph{\setlength{\rightskip}{0pt plus 5cm}int highnumpart = 16}\hfill



Definition at line 77 of file main.C.\label{main.C_a49}
\index{main.C@{main.C}!initial_cooling@{initial\_\-cooling}}
\index{initial_cooling@{initial\_\-cooling}!main.C@{main.C}}
\paragraph{\setlength{\rightskip}{0pt plus 5cm}double initial\_\-cooling = INITIAL\_\-COOLING}\hfill



Definition at line 106 of file main.C.\label{main.C_a48}
\index{main.C@{main.C}!initial_temperature@{initial\_\-temperature}}
\index{initial_temperature@{initial\_\-temperature}!main.C@{main.C}}
\paragraph{\setlength{\rightskip}{0pt plus 5cm}double initial\_\-temperature = INITIAL\_\-TEMPERATURE}\hfill



Definition at line 105 of file main.C.\label{main.C_a46}
\index{main.C@{main.C}!layoutname@{layoutname}}
\index{layoutname@{layoutname}!main.C@{main.C}}
\paragraph{\setlength{\rightskip}{0pt plus 5cm}char $\ast$ layoutname = NIL}\hfill



Definition at line 101 of file main.C.\label{main.C_a45}
\index{main.C@{main.C}!library_name@{library\_\-name}}
\index{library_name@{library\_\-name}!main.C@{main.C}}
\paragraph{\setlength{\rightskip}{0pt plus 5cm}char $\ast$ library\_\-name = NIL}\hfill



Definition at line 100 of file main.C.\label{main.C_a39}
\index{main.C@{main.C}!macroMinSize@{macroMinSize}}
\index{macroMinSize@{macroMinSize}!main.C@{main.C}}
\paragraph{\setlength{\rightskip}{0pt plus 5cm}int macro\-Min\-Size = 100}\hfill



Definition at line 92 of file main.C.\label{main.C_a32}
\index{main.C@{main.C}!madonnaAllowRandomMoves@{madonnaAllowRandomMoves}}
\index{madonnaAllowRandomMoves@{madonnaAllowRandomMoves}!main.C@{main.C}}
\paragraph{\setlength{\rightskip}{0pt plus 5cm}int madonna\-Allow\-Random\-Moves = NIL}\hfill



Definition at line 85 of file main.C.\label{main.C_a23}
\index{main.C@{main.C}!madonna_version@{madonna\_\-version}}
\index{madonna_version@{madonna\_\-version}!main.C@{main.C}}
\paragraph{\setlength{\rightskip}{0pt plus 5cm}char $\ast$ madonna\_\-version = "3.2"}\hfill



Definition at line 70 of file main.C.\label{main.C_a28}
\index{main.C@{main.C}!madonnamakeminiplaza@{madonnamakeminiplaza}}
\index{madonnamakeminiplaza@{madonnamakeminiplaza}!main.C@{main.C}}
\paragraph{\setlength{\rightskip}{0pt plus 5cm}int madonnamakeminiplaza = TRUE}\hfill



Definition at line 81 of file main.C.\label{main.C_a29}
\index{main.C@{main.C}!madonnamakepartition@{madonnamakepartition}}
\index{madonnamakepartition@{madonnamakepartition}!main.C@{main.C}}
\paragraph{\setlength{\rightskip}{0pt plus 5cm}int madonnamakepartition = TRUE}\hfill



Definition at line 82 of file main.C.\label{main.C_a42}
\index{main.C@{main.C}!makeChannels@{makeChannels}}
\index{makeChannels@{makeChannels}!main.C@{main.C}}
\paragraph{\setlength{\rightskip}{0pt plus 5cm}int make\-Channels = 0}\hfill



Definition at line 95 of file main.C.\label{main.C_a31}
\index{main.C@{main.C}!maxNumberOfClusterPermutations@{maxNumberOfClusterPermutations}}
\index{maxNumberOfClusterPermutations@{maxNumberOfClusterPermutations}!main.C@{main.C}}
\paragraph{\setlength{\rightskip}{0pt plus 5cm}int max\-Number\-Of\-Cluster\-Permutations = 2}\hfill



Definition at line 84 of file main.C.\label{main.C_a30}
\index{main.C@{main.C}!permutateClustersAtExit@{permutateClustersAtExit}}
\index{permutateClustersAtExit@{permutateClustersAtExit}!main.C@{main.C}}
\paragraph{\setlength{\rightskip}{0pt plus 5cm}int permutate\-Clusters\-At\-Exit = TRUE}\hfill



Definition at line 83 of file main.C.\label{main.C_a35}
\index{main.C@{main.C}!phil_verbose@{phil\_\-verbose}}
\index{phil_verbose@{phil\_\-verbose}!main.C@{main.C}}
\paragraph{\setlength{\rightskip}{0pt plus 5cm}int phil\_\-verbose = 0}\hfill



Definition at line 88 of file main.C.\label{main.C_a27}
\index{main.C@{main.C}!processid@{processid}}
\index{processid@{processid}!main.C@{main.C}}
\paragraph{\setlength{\rightskip}{0pt plus 5cm}int processid}\hfill



Definition at line 80 of file main.C.\label{main.C_a36}
\index{main.C@{main.C}!rand_points@{rand\_\-points}}
\index{rand_points@{rand\_\-points}!main.C@{main.C}}
\paragraph{\setlength{\rightskip}{0pt plus 5cm}int rand\_\-points = 1}\hfill



Definition at line 89 of file main.C.\label{main.C_a19}
\index{main.C@{main.C}!requestedGridPoints@{requestedGridPoints}}
\index{requestedGridPoints@{requestedGridPoints}!main.C@{main.C}}
\paragraph{\setlength{\rightskip}{0pt plus 5cm}int requested\-Grid\-Points[$\,$] = \{NOTINITIALIZED, NOTINITIALIZED, NOTINITIALIZED\}}\hfill



Definition at line 45 of file main.C.\label{main.C_a34}
\index{main.C@{main.C}!set_srand@{set\_\-srand}}
\index{set_srand@{set\_\-srand}!main.C@{main.C}}
\paragraph{\setlength{\rightskip}{0pt plus 5cm}int set\_\-srand = 0}\hfill



Definition at line 87 of file main.C.\label{main.C_a40}
\index{main.C@{main.C}!slicingLayout@{slicingLayout}}
\index{slicingLayout@{slicingLayout}!main.C@{main.C}}
\paragraph{\setlength{\rightskip}{0pt plus 5cm}int slicing\-Layout = 1}\hfill



Definition at line 93 of file main.C.\label{main.C_a26}
\index{main.C@{main.C}!stopquotient@{stopquotient}}
\index{stopquotient@{stopquotient}!main.C@{main.C}}
\paragraph{\setlength{\rightskip}{0pt plus 5cm}int stopquotient = 100}\hfill



Definition at line 78 of file main.C.\label{main.C_a51}
\index{main.C@{main.C}!strng@{strng}}
\index{strng@{strng}!main.C@{main.C}}
\paragraph{\setlength{\rightskip}{0pt plus 5cm}char strng[1+MAXSTR]\hspace{0.3cm}{\tt  [static]}}\hfill



Definition at line 617 of file main.C.\label{main.C_a21}
\index{main.C@{main.C}!thedate@{thedate}}
\index{thedate@{thedate}!main.C@{main.C}}
\paragraph{\setlength{\rightskip}{0pt plus 5cm}char $\ast$ thedate = THEDATE}\hfill



Definition at line 69 of file main.C.\label{main.C_a22}
\index{main.C@{main.C}!thehost@{thehost}}
\index{thehost@{thehost}!main.C@{main.C}}
\paragraph{\setlength{\rightskip}{0pt plus 5cm}char $\ast$ thehost = THEHOST}\hfill



Definition at line 69 of file main.C.