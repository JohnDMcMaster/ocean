\subsection{phil\_\-glob.h File Reference}
\label{phil_glob.h}\index{phil_glob.h@{phil\_\-glob.h}}
{\tt \#include "image.h"}\par
{\tt \#include "routing.h"}\par
\subsubsection*{Compounds}
\begin{CompactItemize}
\item 
struct {\bf \_\-SLICE\_\-INFO}
\end{CompactItemize}
\subsubsection*{Typedefs}
\begin{CompactItemize}
\item 
typedef struct {\bf \_\-SLICE\_\-INFO} {\bf SLICE\_\-INFO}
\item 
typedef struct {\bf \_\-SLICE\_\-INFO} {\bf SLICE\_\-INFO\_\-TYPE}
\item 
typedef struct {\bf \_\-SLICE\_\-INFO}$\ast$ {\bf SLICE\_\-INFOPTR}
\end{CompactItemize}
\subsubsection*{Functions}
\begin{CompactItemize}
\item 
int {\bf phil} (char $\ast$c\-Name,char $\ast$f\-Name,char $\ast$l\-Name,char $\ast$o\-Name, double magn, CIRCUIT $\ast$from\-Partitioner,CIRCUIT $\ast$real\-Circuit=NULL, {\bf GLOBAL\_\-ROUTING} $\ast$g\_\-rout=NULL)
\item 
void {\bf read\-Image\-File} ()
\item 
{\bf IMAGEDESC}$\ast$ {\bf get\-Image\-Desc} ()
\end{CompactItemize}


\subsubsection{Typedef Documentation}
\label{phil_glob.h_a3}
\index{phil_glob.h@{phil\_\-glob.h}!SLICE_INFO@{SLICE\_\-INFO}}
\index{SLICE_INFO@{SLICE\_\-INFO}!phil_glob.h@{phil\_\-glob.h}}
\paragraph{\setlength{\rightskip}{0pt plus 5cm}typedef struct {\bf \_\-SLICE\_\-INFO} SLICE\_\-INFO}\hfill

\label{phil_glob.h_a5}
\index{phil_glob.h@{phil\_\-glob.h}!SLICE_INFOPTR@{SLICE\_\-INFOPTR}}
\index{SLICE_INFOPTR@{SLICE\_\-INFOPTR}!phil_glob.h@{phil\_\-glob.h}}
\paragraph{\setlength{\rightskip}{0pt plus 5cm}typedef struct {\bf \_\-SLICE\_\-INFO}$\ast$ SLICE\_\-INFOPTR}\hfill

\label{phil_glob.h_a4}
\index{phil_glob.h@{phil\_\-glob.h}!SLICE_INFO_TYPE@{SLICE\_\-INFO\_\-TYPE}}
\index{SLICE_INFO_TYPE@{SLICE\_\-INFO\_\-TYPE}!phil_glob.h@{phil\_\-glob.h}}
\paragraph{\setlength{\rightskip}{0pt plus 5cm}typedef struct {\bf \_\-SLICE\_\-INFO} SLICE\_\-INFO\_\-TYPE}\hfill



\subsubsection{Function Documentation}
\label{phil_glob.h_a2}
\index{phil_glob.h@{phil\_\-glob.h}!getImageDesc@{getImageDesc}}
\index{getImageDesc@{getImageDesc}!phil_glob.h@{phil\_\-glob.h}}
\paragraph{\setlength{\rightskip}{0pt plus 5cm}{\bf IMAGEDESC} $\ast$ get\-Image\-Desc (void)}\hfill



Definition at line 160 of file phil.C.\small\begin{verbatim}00163 {
00164   return parInt.getImageDesc();
00165 
00166 }
\end{verbatim}\normalsize 
\label{phil_glob.h_a0}
\index{phil_glob.h@{phil\_\-glob.h}!phil@{phil}}
\index{phil@{phil}!phil_glob.h@{phil\_\-glob.h}}
\paragraph{\setlength{\rightskip}{0pt plus 5cm}int phil (char $\ast$ {\em c\-Name}, char $\ast$ {\em f\-Name}, char $\ast$ {\em l\-Name}, char $\ast$ {\em o\-Name}, double {\em magn}, CIRCUIT $\ast$ {\em from\-Partitioner}, CIRCUIT $\ast$ {\em real\-Circuit} = NULL, {\bf GLOBAL\_\-ROUTING} $\ast$ {\em glob\_\-rout} = NULL)}\hfill



Definition at line 45 of file phil.C.

Referenced by {\bf madonna}().\small\begin{verbatim}00049 {
00050     
00051     cout << "\n------ Detailed placement of (" << oName << "(" << cName 
00052          << "(" << fName << "(" << lName << "))))...\n";
00053     
00054     if(doTranAna<0)
00055       doTranAna=0;
00056 
00057     if( doTranAna==1)
00058     {
00059       slicingLayout=1;
00060       doTranAna=1;
00061       rand_points=1;
00062     }
00063 
00064     
00065     plcm = new Plcm(cName,fName,lName,
00066                      parInt.getImageDesc(),
00067              parInt.getImageMap(),
00068              oName,
00069              fromPartitioner,
00070              (makeChannels==1?glob_rout:NULL),
00071              Boolean(slicingLayout),
00072              Boolean(doTranAna),
00073                      Boolean(phil_verbose),
00074                      Boolean(rand_points),set_srand,macroMinSize);
00075 
00076     cout << "------ Reading input data" << endl;
00077     
00078     plcm-> read();    
00079     
00080     cout << "------ Creating temporary data structures" << endl;
00081     
00082     plcm->prepare();
00083     
00084     int done;
00085 
00086     do
00087     {
00088       cout << "------ Placement plane size : [" << requestedGridPoints[HOR] << ":" 
00089       << requestedGridPoints[VER] << "]" << endl;
00090 
00091       plcm->createPlane(requestedGridPoints[HOR],requestedGridPoints[VER]);
00092 
00093       cout << "------ Doing placement " << endl;
00094 
00095       done = ! plcm->placement();
00096 
00097       if(!done)         // will have to increase the size of the placement
00098       {             // plane
00099 
00100     if(expandableDirection == HOR)
00101         requestedGridPoints[HOR]=int(INC_FACT*requestedGridPoints[HOR]);
00102       else
00103         requestedGridPoints[VER]=int(INC_FACT*requestedGridPoints[VER]);        
00104     plcm->recover();
00105       }
00106     }while(!done);
00107 
00108                 // one small detail - let\'s properly set 
00109                 // the bounding box (it may be smaller than
00110                 // placement plane size)
00111 
00112     plcm->setBbx();
00113 
00114   
00115 
00116     if(doCompresion>0)
00117     {
00118       cout << "------ Compacting  " << endl;    
00119       plcm->compaction();
00120 
00121     }
00122 
00123     if(doTranAna)
00124     {
00125       cout << "------ Transparencies Analysis. " << endl;
00126       plcm->doTranspAnalysis();
00127     }
00128 
00129 
00130 
00131 #ifdef __MSDOS__
00132     fclose(stdprn);
00133     fclose(stdaux);
00134 #endif
00135 //    if(!doTranAna)        // it\'\' be passed to the global router
00136     if (1)
00137     {               // so don\'t write it yet.
00138       cout << "------ Writing created layout to database" << endl;
00139       plcm->write(realCircuit);
00140     }
00141     else
00142       plcm->write(realCircuit,false); // false == do not write
00143 
00144     delete plcm;
00145     
00146     return 0;
00147 }
\end{verbatim}\normalsize 
\label{phil_glob.h_a1}
\index{phil_glob.h@{phil\_\-glob.h}!readImageFile@{readImageFile}}
\index{readImageFile@{readImageFile}!phil_glob.h@{phil\_\-glob.h}}
\paragraph{\setlength{\rightskip}{0pt plus 5cm}void read\-Image\-File (void)}\hfill



Definition at line 150 of file phil.C.\small\begin{verbatim}00153 {
00154   parInt.read();
00155 
00156 }
\end{verbatim}\normalsize 
